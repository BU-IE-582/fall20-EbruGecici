% Options for packages loaded elsewhere
\PassOptionsToPackage{unicode}{hyperref}
\PassOptionsToPackage{hyphens}{url}
%
\documentclass[
]{article}
\usepackage{lmodern}
\usepackage{amssymb,amsmath}
\usepackage{ifxetex,ifluatex}
\ifnum 0\ifxetex 1\fi\ifluatex 1\fi=0 % if pdftex
  \usepackage[T1]{fontenc}
  \usepackage[utf8]{inputenc}
  \usepackage{textcomp} % provide euro and other symbols
\else % if luatex or xetex
  \usepackage{unicode-math}
  \defaultfontfeatures{Scale=MatchLowercase}
  \defaultfontfeatures[\rmfamily]{Ligatures=TeX,Scale=1}
\fi
% Use upquote if available, for straight quotes in verbatim environments
\IfFileExists{upquote.sty}{\usepackage{upquote}}{}
\IfFileExists{microtype.sty}{% use microtype if available
  \usepackage[]{microtype}
  \UseMicrotypeSet[protrusion]{basicmath} % disable protrusion for tt fonts
}{}
\makeatletter
\@ifundefined{KOMAClassName}{% if non-KOMA class
  \IfFileExists{parskip.sty}{%
    \usepackage{parskip}
  }{% else
    \setlength{\parindent}{0pt}
    \setlength{\parskip}{6pt plus 2pt minus 1pt}}
}{% if KOMA class
  \KOMAoptions{parskip=half}}
\makeatother
\usepackage{xcolor}
\IfFileExists{xurl.sty}{\usepackage{xurl}}{} % add URL line breaks if available
\IfFileExists{bookmark.sty}{\usepackage{bookmark}}{\usepackage{hyperref}}
\hypersetup{
  pdftitle={Interesting R Applications},
  pdfauthor={Ebru Gecici},
  hidelinks,
  pdfcreator={LaTeX via pandoc}}
\urlstyle{same} % disable monospaced font for URLs
\usepackage[margin=1in]{geometry}
\usepackage{graphicx,grffile}
\makeatletter
\def\maxwidth{\ifdim\Gin@nat@width>\linewidth\linewidth\else\Gin@nat@width\fi}
\def\maxheight{\ifdim\Gin@nat@height>\textheight\textheight\else\Gin@nat@height\fi}
\makeatother
% Scale images if necessary, so that they will not overflow the page
% margins by default, and it is still possible to overwrite the defaults
% using explicit options in \includegraphics[width, height, ...]{}
\setkeys{Gin}{width=\maxwidth,height=\maxheight,keepaspectratio}
% Set default figure placement to htbp
\makeatletter
\def\fps@figure{htbp}
\makeatother
\setlength{\emergencystretch}{3em} % prevent overfull lines
\providecommand{\tightlist}{%
  \setlength{\itemsep}{0pt}\setlength{\parskip}{0pt}}
\setcounter{secnumdepth}{-\maxdimen} % remove section numbering

\title{Interesting R Applications}
\author{Ebru Gecici}
\date{28 10 2020}

\begin{document}
\maketitle

{
\setcounter{tocdepth}{3}
\tableofcontents
}
\hypertarget{some-news-about-r}{%
\section{\texorpdfstring{Some News About
\emph{R}}{Some News About R}}\label{some-news-about-r}}

\hypertarget{multi-stage-financial-modeling-using-machine-learning-in-r}{%
\subsection{Multi-stage Financial Modeling using Machine Learning in
R}\label{multi-stage-financial-modeling-using-machine-learning-in-r}}

In the R Conference, which is accessed on YouTube, one of the project
participants provides information about multi-stage financial modeling
in R. The package focuses on commodity prices such as corn, cotton, and
soybean, which are affected by many factors like weather, economy, etc.
Moreover, these data can be obtained via different sources and they can
be from time horizon, e.g.~daily, monthly, and weekly. To read, merge,
clean, and visualize this data financial modeling package can be used.
The video provides a simple implementation of the financial modeling
package by using soybean prices. Moreover, it provides an example of a
model which is obtained by using R programming. This model is
transformed into a dashboard, which is a kind of decision support
system. This system provides visualization and trial for financial
modeling.

Detailed information can be obtained from the
\href{https://www.youtube.com/watch?v=-cgwDHzd0p4}{YouTube}

\hypertarget{an-r-view-into-epidemiology}{%
\subsection{An R View into
Epidemiology}\label{an-r-view-into-epidemiology}}

Today, the most important issue in health care is Covid-19, which is a
pandemic with high infectivity around the world. Even though many people
take into account the prevention to protect themselves, these
protections are not enough to disappear from the virus. For this reason,
people check the daily data news about the pandemic and they estimate
the future of the spread of this virus. The blog, which is written by
\emph{Joseph Rickert}, mentions that the comparison of the model about
pandemic can be difficult, i.e., especially you are not an expert in
this area. For this reason, in this post, the author provides useful
materials for people, who are R literate or interested in the R
programming language. In other words, there are packages, which can be
useful to make an analysis about the epidemic.

The essay, \emph{An R view into Epidemiology}, provides information
about packages about epidemics. According to the results, there are
nearly a hundred packages that take place in the R. Note that these
packages information can be obtained by using \textbf{pkgserach} and
\textbf{dlstat} functions, which give information about packages and
download information about the packages, respectively. Six of epidemic
packages are \textbf{DSAIDE}, \textbf{epicontacts}, \textbf{EpiEstim},
\textbf{EpiModel}, \textbf{epitrix}, \textbf{surveillance}. This post
provides information about these packages. Most of these packages can be
used by experts,on the other hand, the documentation of these packages
is understandable to both experts, students, and people who are
interested in epidemics and R.

For more information, you can visit the
\href{https://rviews.rstudio.com/2020/05/20/some-r-resources-for-epidemiology/}{blog
page}

\hypertarget{analyzing-data-from-covid19-r-package}{%
\subsection{Analyzing Data from Covid19 R
Package}\label{analyzing-data-from-covid19-r-package}}

Another Covid19 related study, which is obtained by using R programming,
is posted by R-bloggers. In this post, the Covid19 R package is
introduced and some implementation with this package is presented. The
used data is collected from the Human Mortality Database. To make
analyzing \textbf{covid19} function is used. This function contains more
information, for this reason, to make an analysis some parts of this
information, i.e., country, date, population, and death parameters, are
used. After this filtering process, the data is visualized by using
\textbf{ggplot} function. Then to find the most affected countries, the
query was addressed in the post.

To see the detail of the R implementation, you can visit
\href{https://www.r-bloggers.com/analyzing-data-from-covid19-r-package/}{blog
page}

\hypertarget{covid-epidemiology-with-r}{%
\subsection{Covid Epidemiology with R}\label{covid-epidemiology-with-r}}

Another Covid and R programming related post is presented by \emph{Tim
Churches}. In the R CRAN, there are some packages related to the
epidemics. By using these packages, data can be rearranged, and by using
this data some predictions and calculations can be made. In this post,
US Covid data from Johns Hopkins University and Wikipedia (by using
\emph{rvest} package, i.e., part of \emph{tidyverse}) is rearranged. By
using \textbf{earlyR} and \textbf{EpiEstim} packages, these data are
analyzed. That is, \textbf{get\_R()} function is used to calculate the
maximum likelihood estimate for reproduction number, and
\textbf{overall\_infectivity()} function in EpiEstim calculates the
infectivity of the pandemic. These values also can be visualized by
using ggplot2 function. However, this calculation is only one part of
the study. For this process, the incidence package is used to fit the
log-linear model. By using this package, two models can be created: the
growth phase and the decay phase. In the current situation, the pandemic
is in the growth phase, for this reason in the post only growth phase
model is presented. Then, by using this model, the future of the
pandemic can be estimated.

In shorts, the Covid data can be visualized, analyzed, modeled, and
estimated by using R packages and functions. By using these packages,
useful decision tools can be created to give information about Covid19.

To get more information about the R implementation in Covid19, you can
visit
\href{https://rviews.rstudio.com/2020/03/05/covid-19-epidemiology-with-r/}{blog
page}

\hypertarget{multi-stage-financial-modeling-using-machine-learning-in-r-1}{%
\subsection{Multi-stage Financial Modeling using Machine Learning in
R}\label{multi-stage-financial-modeling-using-machine-learning-in-r-1}}

In the R Conference, which is accessed on YouTube, one of the project
participants provides information about multi-stage financial modeling
in R. The package focuses on commodity prices such as corn, cotton, and
soybean, which are affected by many factors like weather, economy, etc.
Moreover, these data can be obtained via different sources and they can
be from time horizon, e.g.~daily, monthly, and weekly. To read, merge,
clean, and visualize this data financial modeling package can be used.
The video provides a simple implementation of the financial modeling
package by using soybean prices. Moreover, it provides an example of a
model which is obtained by using R programming. This model is
transformed into a dashboard, which is a kind of decision support
system. This system provides visualization and trial for financial
modeling.

Detailed information can be obtained from the
\href{https://www.youtube.com/watch?v=-cgwDHzd0p4}{YouTube}

\emph{Note: This page is made by using
\href{https://pjournal.github.io/boun01-EbruGecici/}{previous course
example}}

\end{document}
